\section{I* Framework}
Public libraries handle large amount of information each day. Patrons come and go and so do books. There is a need to visualize this data in order to take actions. Administrators are interested to have reports and summaries regarding the books.

\subsection{Actors}
\subsubsection{Librarian}
A person in charge of managing the books. She lends books to patrons and also receives returned books. She has a book to keep a record of lendings and returnings of books.

The librarian asks for a patron id when lending or accepting the return of a book. Sometimes patrons do not return a book on time. When lending a book the librarian annotates the deadline, if the patron hasn't returned the book by then that means a penalty will be given it could be monetary or another kind.

\subsubsection{Patron}
A patron is a person interesed in borrowing a book. Once in the library, sometimes there is little machine where you can search for your book and it will output which you can use to borrow a book, as long as you have your patron id. One can go to the registrar office to get a patron id, there you will be asked for your citizen id and proof of where you live.

\subsubsection{Administrator}
The Administrator is in charge of the library. Administering a library is no easy task, books come and go. It helps a lot to have reports of the lent and returned books.

\subsubsection{Registrar office}
The registrar office registers new patrons. It is required a proof of residence as well as the citizen's ID. 

The office provides users of patron cards which they can use to borrow and return books.

